\documentclass[11pt,a4paper]{ltxdoc} 
\usepackage[spanish,es-noindentfirst,es-tabla]{babel}

\usepackage[utf8]{inputenc}
\usepackage[T1]{fontenc}
\usepackage{graphicx}
\usepackage{float}
\usepackage[margin=2.5cm,left=3.5cm]{geometry}

\usepackage{mathpazo}
\usepackage{longtable}
\usepackage{aleph-comandos}

\setlength{\parskip}{0.2\baselineskip}
\renewcommand{\baselinestretch}{1.1}

\newcommand{\file}[1]{\texttt{#1}}
\newcommand{\option}[1]{\texttt{#1}}
\newcommand{\package}[1]{\texttt{#1}}
\newcommand{\ejemplo}[2]{\fbox{#1}\hspace{1cm}\fbox{$#2$}}

\title{\file{aleph-moodle.sty}}
\author{Proyecto Alephsub0\\ Jonathan Ortiz y Andr\'es Merino}
\date{2020-08-17\\ Versión 1.0}

\usepackage[colorlinks,linkcolor=teal,urlcolor=teal,
   citecolor=black,bookmarks=true]{hyperref}
\usepackage{url}

\begin{document}
 
\maketitle
 
\begin{abstract}
    \file{aleph-moodle.sty} es un paquete creado para utilizar los comandos del paquete \file{aleph-comandos.sty} en la generación de bancos de preguntas que serán exportados a Moodle mediante el paquete \file{moodle.sty}. Esta paquete fue generada dentro del proyecto Alephsub0 (\url{https://www.alephsub0.org/}).
\end{abstract}

\section{Introducción}

El paquete \file{aleph-moodle.sty} es generado por Jonathan Ortiz y traduce los comandos definidos en \file{aleph-comandos.sty} a su versión del paquete \file{moodle.sty}.



\section{Uso y opciones}

Para cargar la clase se utiliza: \cs{usepackage}|{aleph-moodle}|.
Con esto, se generará un archivo |xml| con el banco de preguntas para subir a Moodle. En caso de usar Overleaf, se puede acceder al archivo en los archivos generados del proyecto; para acceder a los archivos generados, ver este enlace \href{https://www.overleaf.com/learn/how-to/View_generated_files}{este enlace}.

Para visualizar un ejemplo puedes acceder al repositorio de GitHub de este paquete (clic \href{https://github.com/mate-andres/LaTeX_aleph-moodle}{aquí}) o buscarlo en la galería de plantilla de Overleaf (clic \href{https://www.overleaf.com/latex/templates/ejemplo-de-banco-de-preguntas-para-exportar-a-moodle/nxkrctpyvzhn}{aquí}).
    

\DescribeMacro{draft} 
    Este paquete tiene una única opción: |draft|, la cual es pasada al paquete |moodle.sty|. Cuando esta opción es incluida no se genere el archivo |xml| (ver docuemntación del paquete moodle).


\section{Comandos}

Los comandos proporcionados por el presente paquete son los mismo que los del paquete |aleph-comando.sty|, únicamente se generan las definiciones necesarias para tener compatibilidad con el paquete |moodle.sty|.

\section{Alcances}

Con este paquete se pueden utilizar todos los comandos propios de \LaTeX\ y del paquete |aleph-comando.sty|, salvo

\begin{itemize}
    \item \cs{suc}
    \item \cs{comentario}
    \item \cs{Mat} siempre tiene cuerpo los reales $\R$.
\end{itemize}

Además, permite utilizar la mayoría de las opciones de la plataforma Moodle: Elección de varias respuestas correctas, presentar retroalimentación en cada pregunta, ingreso de respuestas númericas, utilización de figuras, etc.


\section{Limitaciones}

La conversión del archivo |tex| a formato |xml| (moodle) no es posible cuando existen tildes, eñes, doble barra invertida |\\|, signo de abrir interrogación |?`|, ni llaves. De necesitarse la escritura de estos caracteres se debe reemplazarlos de la siguiente manera:

\begin{center}
    \begin{tabular}{cc}
    \hline
        Comando & Acción \\
    \hline
        |\'a| & á \\
       |\'e| & é \\
       |\'i| & í \\
       |\'o| & ó \\
       |\'u| & ú \\
       |\~n| & \~n \\
       |\cr| & |\\| \\
       |\lbrace | $\cdot$ |\rbrace|
       &$\lbrace \cdot \rbrace$\\
        |&iquest| & ?`\\
    \hline
    \end{tabular}
\end{center}

Además, como el ambiente |aling*| es matemático en LaTeX, al momento de realizar la conversión a Moodle, no lo reconoce como comando matemático. La forma de evitar esto es utilizar el ambiente |aligned|, el cual se comporta de forma equivalente al |align*|, pero este debe ser utilizado dentro de un ambiente matemático como \$ o corchetes. Esto se debe hacer con cada ambiente que sea estrictamente matemático.

Al reemplazar ?` por |&iquest|, se produce un error al compilar en \LaTeX, este error no tiene efecto al generar el archivo |xml| y todo marcha bien al subir el archivo a Moodle.

Para adjuntar figuras, el archivo con la figura debe estar dentro de la misma carpeta que está el archivo principal sino, a pesar de no haber errores al compilar, en el Moodle no se cargará la figura.

\section{Sugerencias}

La mejor forma de realizar cuestionarios para Moodle mediante la utilización del paquete |aleph-moodle.sty| es escribir de manera normal todo el cuestionario en modo |draft| (\cs{usepackage}|[draft]{aleph-moodle}|) y al finalizar su revisión: buscar-reemplazar todos las tildes, eñes, doble barra invertidas y ?' por sus comandos respectivos.

Para adjuntar imágenes, todo funciona muy bien si utilizamos formato .png y dentro de |\includegraphics[]{}|, recordando que las únicas opciones habilitadas para este es solo ancho y alto.

Para una explicación completa de las demás opciones que se pueden utilizar con este paquete se puede consultar a \url{https://ctan.org/pkg/moodle}.

Cualquier problema, por favor reportarlo a\\
jonathan.ortiz.c347@gmail.com

\newpage
\DocInput{aleph-moodle.dtx}

\end{document}