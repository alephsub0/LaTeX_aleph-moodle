\documentclass[11pt,a4paper]{ltxdoc} 
\usepackage[spanish,es-noindentfirst,es-tabla]{babel}

\usepackage[utf8]{inputenc}
\usepackage[T1]{fontenc}
\usepackage{graphicx}
\usepackage{float}
\usepackage[margin=2.5cm,left=3.5cm]{geometry}
\usepackage{changelog}

\usepackage{mathpazo}
\usepackage{longtable}
\usepackage{aleph-comandos}

\setlength{\parskip}{0.2\baselineskip}
\renewcommand{\baselinestretch}{1.1}

\newcommand{\file}[1]{\texttt{#1}}
\newcommand{\option}[1]{\texttt{#1}}
\newcommand{\package}[1]{\texttt{#1}}
\newcommand{\ejemplo}[2]{\fbox{#1}\hspace{1cm}\fbox{$#2$}}

\title{\file{aleph-moodle.sty}}
\author{Proyecto Alephsub0\\ Jonathan Ortiz y Andr\'es Merino}
\date{2023-12-25\\ Versión 2.0}

\usepackage[colorlinks,linkcolor=teal,urlcolor=teal,
   citecolor=black,bookmarks=true]{hyperref}
\usepackage{url}

\begin{document}
 
\maketitle
 
\begin{abstract}
    \file{aleph-moodle.sty} es un paquete que implementa varias modificaciones el funcionamiento del paquete |moodle|. Esta paquete fue generada dentro del proyecto Alephsub0 (\url{https://www.alephsub0.org/}).
\end{abstract}

\section{Introducción}

El paquete \file{aleph-moodle.sty} es generado por Jonathan Ortiz e implementa varias modificaciones al paquete |moodle.sty|.

\section{Modificaciones}

Las modificaciones implementadas al paquete |moodle.sty| son las siguientes:

\begin{enumerate}
    \item Formato retroalimentación
    \item Cambio en los estilos de enumeración del tipo de pregunta |multi|
    \item Cambio de formato de la retroalimentación numérica
    \item Cambio de formato de la retroalimentación de respuesta corta
    \item Eliminación del total de puntos
    \item Eliminación de las etiquetas en el tipo multiopción
    \item Eliminación de las etiquetas en el tipo numérico
    \item Corrección del comando pm
\end{enumerate}

Para una mejor comprensión de las modificaciones, se recomienda consultar la sección de implementación.

\section{Uso y opciones}

Para cargar la clase se utiliza: \cs{usepackage}|{aleph-moodle}|. Para la generación del archivo |xml| el motor de compilación debe ser XeLaTeX o LuaLaTeX. Con esto, se obtiene el banco de preguntas para subir a Moodle. En el caso de requerir únicamente el archivo |pdf|, se debe utilizar pdfLaTeX.

En caso de usar Overleaf, se puede acceder al archivo en los archivos generados del proyecto; para acceder a los archivos generados, ver este enlace \href{https://www.overleaf.com/learn/how-to/View_generated_files}{este enlace}.

Para visualizar un ejemplo puedes acceder al repositorio de GitHub de este paquete (clic \href{https://github.com/mate-andres/LaTeX_aleph-moodle}{aquí}) o buscarlo en la galería de plantilla de Overleaf (clic \href{https://www.overleaf.com/latex/templates/ejemplo-de-banco-de-preguntas-para-exportar-a-moodle/nxkrctpyvzhn}{aquí}).
    
\subsection{Paquete \texttt{aleph-comandos.sty}}

Para el uso de los comandos definidos en el paquete |aleph-comandos.sty|, se debe cargar el paquete |aleph-comandos.sty| antes de cargar el paquete |aleph-moodle.sty| y además se debe utilizar el comando |\moodleregisternewcommands| entre la carga de ambos paquetes.

\begin{verbatim}
    \usepackage{aleph-moodle}
    \moodleregisternewcommands
    \usepackage{aleph-comandos}
\end{verbatim}

\section{Limitaciones}

A partir de la versión 0.6a del paquete |moodle.sty| es posible utilizar XeLaTeX para la compilación, esto permite el uso nativo de tildes y eñes, etc. No obstante en el caso de compilar con XeLaTeX o LuaLaTeX el archivo |pdf|, presenta problemas con las fuentes y caracteres especiales en el pdf.

Para una mejor comprensión de las limitaciones de este paquete, se recomienda consultar \url{https://ctan.org/pkg/moodle}.

\section{Sugerencias}

Para adjuntar imágenes, todo funciona muy bien si utilizamos formato .png y dentro de |\includegraphics[]{}|, recordando que las únicas opciones habilitadas para este es solo ancho y alto.

Para una explicación completa de las demás opciones que se pueden utilizar con este paquete se puede consultar a \url{https://ctan.org/pkg/moodle}.

Cualquier problema, por favor reportarlo a\\
jonathan.ortiz.c347@gmail.com

\begin{changelog}[author=Andés Merino \& Jonathan Ortiz,
    sectioncmd=\section]
    % version 2.0
    \begin{version}[author=Andés Merino \& Jonathan Ortiz ,v=2.0,
        date=2023-11-27]
        \removed
        \item Se eliminan las adaptaciones a los comandos de \texttt{aleph-comandos.sty} debido a que ya no son necesarias.
        \added
        \item Formato retroalimentación
        \item Cambio en los estilos de enumeración del tipo de pregunta |multi|
        \item Cambio de formato de la retroalimentación numérica
        \item Cambio de formato de la retroalimentación de respuesta corta
        \item Eliminación del total de puntos
        \item Eliminación de las etiquetas en el tipo multiopción
        \item Eliminación de las etiquetas en el tipo numérico
        \item Corrección del comando pm
        \changed
        \item Se modifica el comportamiento de la compilación de acuerdo al motor de compilación usado.
    \end{version}
    % version 1.0
    \shortversion{v=1.0,
    date=2020-01-07,
    changes=Primera versión del paquete \texttt{aleph-moodle}}
    % version 0.1
    \shortversion{v=0.1,
    date=2018-08-06,
    changes=Última versión del paquete \texttt{mooodleEPN}}
\end{changelog}

\newpage
\DocInput{aleph-moodle.dtx}

\end{document}