\documentclass[a4,11pt]{aleph-notas}
% Actualizado en febrero de 2024
% Funciona con TeXLive 2022
% Para obtener solo el pdf, compilar con pdfLaTeX. 
% Para obtener el xml compilar con XeLaTeX.

% -- Paquetes adicionales
\usepackage{aleph-moodle}
\moodleregisternewcommands
% Todos los comandos nuevos deben ir luego del comando anterior
\usepackage{aleph-comandos}

% -- Datos 
\institucion{Proyecto Alephsub0}
% \carrera{}
\asignatura{Álgebra lineal}
\tema{Banco de preguntas no. 01: Determinantes}
\autor{Mat. Andrés Merino}
\fecha{Semestre 2022-2}
% \logouno[0.14\textwidth]{logoPUCE_04_ac}
\definecolor{colortext}{HTML}{0030A1}
\definecolor{colordef}{HTML}{00A1DE}
\fuente{montserrat}
% \fuente{mathpazo}

% -- Otros comandos
\DeclareMathOperator{\sen}{sen}



\begin{document}

\encabezado

%%%%%%%%%%%%%%%%%%%%%%%%%%%%%%%%%%%%%%%%
\section{Banco de preguntas}
%%%%%%%%%%%%%%%%%%%%%%%%%%%%%%%%%%%%%%%%

%%%%%%%%%%%%%%%%%%%%%%%%%%%%%%%%%%%%%%%%
\begin{quiz}{Preguntas}
%%%%%%%%%%%%%%%%%%%%%%%%%%%%%%%%%%%%%%%%

%%%%%%%%%%%%%%%%%%%%%%%%%%%%%%%%%%%%%%%%
\begin{multi}[%
    % - Retroalimentación
    feedback={La respuesta correcta es $\left[0,+\infty\right[$}
    ]%
    % - Indentificador
    {Pregunta-01}
    % - Enunciado
    ¿Cuál es la imagen de la función $\func{f}{\R}{\R}$ tal que $x\mapsto x^2$?
    \item $\R$
    \item* $\left[0,+\infty\right[$
    \item $\left]-\infty,0\right]$
\end{multi}

%%%%%%%%%%%%%%%%%%%%%%%%%%%%%%%%%%%%%%%%
\begin{numerical}[tolerance=0.1,%
    % - Retroalimentación
    feedback={La suma es igual a $2$.}
    ]%
    % - Indentificador
    {Pregunta-02}
    % - Enunciado
    El resultado de $1+1$ es:
    \item 2
\end{numerical}

%%%%%%%%%%%%%%%%%%%%%%%%%%%%%%%%%%%%%%%%
\begin{shortanswer}[%
    % - Retroalimentación
    feedback={El dominio de la relación es $\{a,c,d\}$}
    ]%
    % - Indentificador
    {Pregunta-03}
    % - Enunciado
    Considere la relación $\{(a,b), (a,c), (d,c), (c,c)\}$, el dominio de la relación es (colocar los elementos separados por comas, en orden alfabético y sin espacios):
    \item a,c,d
\end{shortanswer}

%%%%%%%%%%%%%%%%%%%%%%%%%%%%%%%%%%%%%%%%
\begin{multi}[]%
    % - Indentificador
    {Pregunta-06}
    % - Enunciado
    Esta pregunta está hecha para probar algunos caracteres especiales. ¿Pongamos una tabla?
    \[
        \begin{array}{c|c}
            1 & 2 \\\hline
            3 & 4
        \end{array}
    \]
    y una \texttt{aligned}:
    \[
        \begin{aligned}
            a & = b\\
            \lim_{x\to0 }f(x) & = b
        \end{aligned}
    \]
    También incluyamos estos signos (se debe usar $\backslash($ y $\backslash)$): \(2>1\) y \(1<2\). En modo desplegado no hay problema:
    \[
        2>1 \quad\text{y}\quad 1<2. 
    \]
    Finalmente, para operadores definidos por el usuario, se tiene problemas
    un $\text{sen}(x)$
    y 
    un $\text{dom}(F)$.
    \item $(0,1)$
    \item $(0,1)$
    \item* $(0,0)$
\end{multi}

\end{quiz}


%%%%%%%%%%%%%%%%%%%%%%%%%%%%%%%%%%%%%%%%
\begin{quiz}{Tipos de pregunta}
%%%%%%%%%%%%%%%%%%%%%%%%%%%%%%%%%%%%%%%%

%%%%%%%%%%%%%%%%%%%%%%%%%%%%%%%%%%%%%%%%
\begin{multi}[]%
    % - Indentificador
    {multi}
    % - Enunciado
    Which numbers are prime? 
    \item* 2 
    \item* 5 
    \item* 7 
    \item 1 
    \item 6 
\end{multi}



%%%%%%%%%%%%%%%%%%%%%%%%%%%%%%%%%%%%%%%%
\begin{numerical}[tolerance=0.01]%
    % - Indentificador
    {numerical} 
    % - Enunciado
    Approximate value of $\sqrt{2}$? 
    \item[tolerance={1e-1}] 1.4142
    % \item[fraction=20,feedback={twice this!}] 7.0711e-1 
    % \item[fraction=0,feedback={Wrong!}] * 
\end{numerical}


%%%%%%%%%%%%%%%%%%%%%%%%%%%%%%%%%%%%%%%%
\begin{shortanswer}%
    % - Indentificador
    {shortanswer}
    % - Enunciado
    Newton's rival was Gottfried Wilhelm \blank. 
    \item Leibniz 
    \item Leibniz. 
\end{shortanswer}

%%%%%%%%%%%%%%%%%%%%%%%%%%%%%%%%%%%%%%%%
\begin{shortanswer}[usecase]%
    % - Indentificador
    {shortanswer}
    % - Enunciado
    What was Newton's first name? 
    \item Isaac 
    \item[fraction=0,feedback={Simply Isaac!}] Isaa* 
    \item[fraction=0,feedback={This one is Leibniz!}] *Gottfried* 
    \item[fraction=0,feedback={First name, not title!}] Sir* 
    \item[fraction=0,feedback={No\dots}] * 
\end{shortanswer}

%%%%%%%%%%%%%%%%%%%%%%%%%%%%%%%%%%%%%%%%
\begin{essay}[]%
    % - Indentificador
    {essay} 
    % - Enunciado
    Enunciado
    \item information 1 for grader 
\end{essay}

%%%%%%%%%%%%%%%%%%%%%%%%%%%%%%%%%%%%%%%%
\begin{matching}[]%
    % - Indentificador
    {matching} 
    % - Enunciado
    Enunciado
    \item Primero \answer 1
    \item Segundo \answer 2
    \item Tercero \answer 3
\end{matching}

%%%%%%%%%%%%%%%%%%%%%%%%%%%%%%%%%%%%%%%%
\begin{cloze}%
    % - Indentificador
    {cloze} 
    % - Enunciado
    Thanks to calculus, invented by Isaac 
    \begin{shortanswer}[usecase] 
        \item Newton 
    \end{shortanswer}, 
    we know that the derivative of $x^2$ is 
    \begin{multi}[] 
        \item $2x$ 
        \item* $\frac{1}{3} x^3 + C$ 
        \item $0$ 
    \end{multi}
    and that $\int_0^2 x^2\,dx$ equals 
    \begin{numerical} 
        \item[tolerance={4e-4}] 2.667 
    \end{numerical}. 
    Thanks, Isaac! 
\end{cloze}


\end{quiz}


\end{document}