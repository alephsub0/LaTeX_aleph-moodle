\documentclass[a4,11pt]{aleph-notas-beta}
% Para obtener el pdf, compilar solo con pdfLaTeX. 
% Para obtener el xml compilar con XeLaTeX o LuaLaTeX.


% -- Paquetes adicionales
\usepackage{aleph-moodle-beta}
\usepackage{aleph-comandos}

% -- Datos 
\institucion{Proyecto Alephsub0}
% \carrera{}
\asignatura{Álgebra lineal}
\tema{Banco de preguntas no. 01: Determinantes}
\autor{Mat. Andrés Merino}
\fecha{Semestre 2022-2}

% \logouno[0.14\textwidth]{logoPUCE_04_ac}
\definecolor{colortext}{HTML}{0030A1}
\definecolor{colordef}{HTML}{00A1DE}
\fuente{montserrat}

% -- Otros comandos




\begin{document}

\encabezado

\vspace*{-8mm}
%%%%%%%%%%%%%%%%%%%%%%%%%%%%%%%%%%%%%%%%
\section{Indicaciones}
%%%%%%%%%%%%%%%%%%%%%%%%%%%%%%%%%%%%%%%%

Se plantea una variable con sus características y se solicita identificar el tipo al que pertenece.

%%%%%%%%%%%%%%%%%%%%%%%%%%%%%%%%%%%%%%%%
\section{Banco de preguntas}
%%%%%%%%%%%%%%%%%%%%%%%%%%%%%%%%%%%%%%%%

%%%%%%%%%%%%%%%%%%%%%%%%%%%%%%%%%%%%%%%%
\begin{quiz}{Preguntas-OpMult}
%%%%%%%%%%%%%%%%%%%%%%%%%%%%%%%%%%%%%%%%

%%%%%%%%%%%%%%%%%%%%%%%%%%%%%%%%%%%%%%%%
\begin{multi}[]%
    % - Indentificador
    {NReales-OpMult-01}
    % - Enunciado
    Dados $x\in \R$, si $x^2=4$ entonces:
    \item $x=2$
    \item $x=-2$
    \item* $\pm 2$
    \item Ninguna respuesta
\end{multi}



\end{quiz}




\end{document}