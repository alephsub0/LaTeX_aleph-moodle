\documentclass[a4,11pt]{aleph-notas}
% Para obtener solo el pdf, compilar con pdfLaTeX. 
% Para obtener el xml compilar con XeLaTeX.

% -- Paquetes adicionales
\usepackage{aleph-moodle}
\moodleregisternewcommands
\usepackage{aleph-comandos}

% -- Datos 
\institucion{Proyecto Alephsub0}
% \carrera{}
\asignatura{Álgebra lineal}
\tema{Banco de preguntas no. 01: Determinantes}
\autor{Mat. Andrés Merino}
\fecha{Semestre 2022-2}
% \logouno[0.14\textwidth]{logoPUCE_04_ac}
\definecolor{colortext}{HTML}{0030A1}
\definecolor{colordef}{HTML}{00A1DE}
\fuente{montserrat}
% \fuente{mathpazo}

% -- Otros comandos




\begin{document}

\encabezado

\vspace*{-8mm}
%%%%%%%%%%%%%%%%%%%%%%%%%%%%%%%%%%%%%%%%
\section{Indicaciones}
%%%%%%%%%%%%%%%%%%%%%%%%%%%%%%%%%%%%%%%%

Se plantea una variable con sus características y se solicita identificar el tipo al que pertenece.

%%%%%%%%%%%%%%%%%%%%%%%%%%%%%%%%%%%%%%%%
\section{Banco de preguntas}
%%%%%%%%%%%%%%%%%%%%%%%%%%%%%%%%%%%%%%%%

%%%%%%%%%%%%%%%%%%%%%%%%%%%%%%%%%%%%%%%%
\begin{quiz}{Preguntas}
%%%%%%%%%%%%%%%%%%%%%%%%%%%%%%%%%%%%%%%%

%%%%%%%%%%%%%%%%%%%%%%%%%%%%%%%%%%%%%%%%
\begin{multi}[%
    % - Retroalimentación
    feedback={La respuesta correcta es $\left[0,+\infty\right[$}
    ]%
    % - Indentificador
    {Pregunta-01}
    % - Enunciado
    ¿Cuál es la imagen de la función $\func{f}{\R}{\R}$ tal que $x\mapsto x^2$?
    \item $\R$
    \item* $\left[0,+\infty\right[$
    \item $\left]-\infty,0\right]$
\end{multi}

%%%%%%%%%%%%%%%%%%%%%%%%%%%%%%%%%%%%%%%%
\begin{numerical}[tolerance=0.1,%
    % - Retroalimentación
    feedback={La suma es igual a $2$.}
    ]%
    % - Indentificador
    {Pregunta-02}
    % - Enunciado
    El resultado de $1+1$ es:
    \item 2
\end{numerical}

%%%%%%%%%%%%%%%%%%%%%%%%%%%%%%%%%%%%%%%%
\begin{shortanswer}[%
    % - Retroalimentación
    feedback={El dominio de la relación es $\{a,c,d\}$}
    ]%
    % - Indentificador
    {Pregunta-03}
    % - Enunciado
    Considere la relación $\{(a,b), (a,c), (d,c), (c,c)\}$, el dominio de la relación es (colocar los elementos separados por comas, en orden alfabético y sin espacios):
    \item a,c,d
\end{shortanswer}


%%%%%%%%%%%%%%%%%%%%%%%%%%%%%%%%%%%%%%%%
\begin{multi}[%
    % - Retroalimentación
    feedback={Para cumplir la restricción del problema, necesitamos que $d((a,1),(1,-1)) = d((a,1),(-1,1))$, es decir, necesitamos que $\mathbb{R}$
    \[
        \sqrt{(a-1)^2+(1-(-1))^2}=\sqrt{(a-(-1))^2+(1-1)^2},
    \]
    cuya solución es $a=1$.}]%
    % - Indentificador
    {Pregunta-04}
    % - Enunciado
    Considere la \textbf{transformación lineal} definida por
    \[
        T(x) = \{2x_1,x_1+x_2\}.
    \]
    ¿Cuál de los siguientes vectores pertenece al núcleo de $T$?
    \item $(1,-1)$
    \item $(2,-2)$
    \item $(0,1)$
    \item* $(0,0)$
\end{multi}


%%%%%%%%%%%%%%%%%%%%%%%%%%%%%%%%%%%%%%%%
\begin{multi}[]%
    % - Indentificador
    {Pregunta-06}
    % - Enunciado
    Esta pregunta está hecha para probar algunos caracteres especiales:  Pongamos una tabla que le haga compañía:
    \[
        \begin{array}{c|c}
            1 & 2 \\\hline
            3 & 4
        \end{array}
    \]
    y una \texttt{aligned}:
    \[
        \begin{aligned}
            a & = b\\
            \lim_{x\to0 }f(x) & = b
        \end{aligned}
    \]
    \item $(1,-1)$
    \item $(2,-2)$
    \item $(0,1)$
    \item* $(0,0)$
\end{multi}

\end{quiz}




\end{document}